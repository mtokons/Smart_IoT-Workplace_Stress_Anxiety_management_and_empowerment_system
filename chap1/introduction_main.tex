\chapter{Introduction}

\textbf{Note that you may have multiple \texttt{{\textbackslash}include} statements here, e.g.\ one for each subsection.}

\section{Motivationss} % why is this a non trivial problem
\blindtext

\begin{table*}\centering
\ra{1.3}
\begin{tabular}{@{}rrrrcrrr@{}}\toprule
& \multicolumn{3}{c}{$w = 8$} & \phantom{abc}& \multicolumn{3}{c}{$w = 16$} \\
\cmidrule{2-4} \cmidrule{6-8} 
& $t=0$ & $t=1$ & $t=2$ && $t=0$ & $t=1$ & $t=2$\\ \midrule
$dir=1$\\
$c$ & 0.0790 & 0.1692 & 0.2945 && 0.3670 & 0.7187 & 3.1815\\
$c$ & -0.8651& 50.0476& 5.9384&& -9.0714& 297.0923& 46.2143\\
$c$ & 124.2756& -50.9612& -14.2721&& 128.2265& -630.5455& -381.0930\\
$dir=0$\\
$c$ & 0.0357& 1.2473& 0.2119&& 0.3593& -0.2755& 2.1764\\
$c$ & -17.9048& -37.1111& 8.8591&& -30.7381& -9.5952& -3.0000\\
$c$ & 105.5518& 232.1160& -94.7351&& 100.2497& 141.2778& -259.7326\\
\bottomrule
\end{tabular}
\caption{A Beautiful and Complex Table}\label{tab:sometable}
\end{table*}

A beautiful table is shown in Table~\ref{tab:sometable}, data from \citet{Ebejer2012} (when citing as part of text, otherwise \citep{Ebejer2012}).

\section{Aims and Objectives} 
\blindtext

\begin{figure}[ht!] % supposedly places it here ...
  \centering
  \includegraphics[width=0.6\linewidth]{test_image_goku}
  \caption[This is the short caption for List of Figures]{A test figure.  This caption is huge, but in the list of figures only the smaller version in the square brackets will appear.\index{Goku il-king}}
  \label{fig:test1}
\end{figure}

A test figure is shown in Figure~\ref{fig:test1}.

\section{Proposed Solution} 

\blindtext
\blindtext

\begin{figure}[!ht]
    \centering
    \subbottom[Goku]{\includegraphics[width=0.3\textwidth]{test_image_goku}}\qquad
    \subbottom[More Goku]{\includegraphics[width=0.3\textwidth]{test_image_goku}}%
    \caption[Short Caption]{The same super saiyan. Two times.}        
    \label{fig:test2}
\end{figure}

Two figures shown side by side are shown in Figure~\ref{fig:test2}.

\subsection{Showing the Use of Acronyms}

In the early nineties, \acs{GSM} was deployed in many European countries. \ac{GSM} offered for the first time international roaming for mobile subscribers. The \acs{GSM}’s use of \ac{TDMA} as its communication standard was debated at length. And every now and then there are big discussion whether \ac{CDMA} should have been chosen over \ac{TDMA}.

If you want to know more about \acf{GSM}, \acf{TDMA}, \acf{CDMA} and other acronyms, just read a book about mobile communication. Just to mention it: There is another \ac{UA}, for testing.


\section{Document Structure}

\blindtext
